\documentclass{article}
\usepackage[utf8]{vietnam}
\usepackage{graphicx}
\usepackage[left=2.00cm, right=2.00cm, top=2.00cm, bottom=2.00]{geometry}
\usepackage{amsmath}
\usepackage{amsfonts}
\usepackage{array}
\usepackage{url}
\title{\textbf{SS004.L1.5}}
\author{Nguyễn Duy Hoàng - 19521537\\Trần Phi Long - 19521803\\Đào Văn Phú - 19522019}
\usepackage{scrextend}
\changefontsizes{11.5pt}
\usepackage{url}\usepackage[colorlinks,urlcolor=blue]{hyperref}
\begin{document}
\begin{center}
    \textbf{\Large{HỢP ĐỒNG THÀNH LẬP NHÓM}}
\end{center}

\begin{flushleft}
\textbf{\large I. Thời gian thành lập nhóm: 12/11/2020}
\newline
\textbf{\large II. Thành viên trong nhóm:}
\begin{center}
    \begin{itemize}
        \item Nguyễn Duy Hoàng - 19521537
        \item Trần Phi Long - 19521803
        \item Đào Văn Phú - 19522019
    \end{itemize}
\end{center}
\textbf{\large III. Mục đích thành lập nhóm}

\begin{itemize}
    \item[-]Nâng cao kỹ năng làm việc nhóm cũng như các kỹ năng mềm của các thành viên trong nhóm.
    \item[-]Cùng nhau hoàn thành tốt các bài tập, nhiệm vụ của môn học đề ra.
    \item[-]Có khả năng sử dụng thành thạo các phần mềm mà môn học yêu cầu.
\end{itemize}
\textbf{\large IV. Vai trò từng thành viên}
\begin{tabular}{|>{\raggedright\arraybackslash}p{3cm}| >{\raggedright\arraybackslash}p{3cm}|
>{\raggedright\arraybackslash}p{3cm}|
>{\raggedright\arraybackslash}p{3cm}|
>{\raggedright\arraybackslash}p{3cm}|}
    \hline
        \begin{center}
            Tên --- Vai trò
        \end{center}

        &Lãnh đạo nhóm và giữ tiến độ

        &Code game (Đồ hoạ và logic)

        &Soạn thảo báo cáo bằng Latex và Tester
        
        &Research and development
        \\
    \hline
        \begin{center}
            Nguyễn Duy Hoàng
        \end{center} & \begin{center}
            X
        \end{center} & \begin{center}
            X (Đồ hoạ)
        \end{center} & \begin{center}
            X
        \end{center} & \begin{center}
            X
        \end{center}
        \\
    \hline
        \begin{center}
            Trần Phi Long
        \end{center} & \begin{center}
            
        \end{center} & \begin{center}
            X (Logic)
        \end{center} & \begin{center}
        
        \end{center} & \begin{center}
            X
        \end{center}
        \\
    \hline
     \begin{center}
            Đào Văn Phú
        \end{center} & \begin{center}
            
        \end{center} & \begin{center}
            
        \end{center} & \begin{center}
            X
        \end{center} & \begin{center}
            X
        \end{center}
        \\
    \hline
\end{tabular}
\textbf{\large V. Chỉ tiêu đánh giá}
\begin{tabular}{|>{\raggedright\arraybackslash}p{4cm}
|>{\raggedright\arraybackslash}p{4cm}
|>{\raggedright\arraybackslash}p{4cm}
|>{\raggedright\arraybackslash}p{4cm}
|>{\raggedright\arraybackslash}p{4cm}|}
    \hline
        Đặc điểm
        & Hoàn thành tốt
        & Hoàn thành
        & Không hoàn thành\\
    \hline
        Thái độ làm việc
        & Hoàn thành tốt nhiệm vụ được giao
        & Hoàn thành nhiệm vụ với sự nhắc nhở
        & Không hoàn thành nhiệm vụ được giao\\
    \hline
        Quản lý thời gian
        & Hoàn thành nhiệm vụ trước thời hạn
        & Hoàn thành đúng thời hạn
        & Không hoàn thành đúng thời hạn
        \\
    \hline
        Giải quyết vấn đề phát sinh
        & Tích cực tìm kiếm giải pháp
        & Nhờ cậy hoặc là dùng sự hỗ trợ nhóm để giải quyết
        & Không tham gia vào vấn đề cần giải quyết
        \\
    \hline
        Nếu ý kiến
        & Sẵn sàng nêu ý kiến.
        & Đưa ra ý kiến một cách bị động
        & Không đưa ra ý kiến nào
        \\
    \hline
        Giữ liên lạc
        & Luôn giữ liên lạc
        & Liên lạc 1 cách thụ động
        & Khó khắn trong việc liên lạc
        \\
    \hline
        
\end{tabular}
\textbf{\large VI. Cam kết}
Sau khi đọc kỹ hợp đồng, các thành viên phải làm theo yêu cầu trong hợp đồng đã đặt ra.
\newpage
\begin{center}
    \textbf{\Large Chữ ký}
\end{center}
\textbf{Nguyễn Duy Hoàng}\\
\textbf{Trần Phi Long}\\
\textbf{Đào Văn Phú}
\end{flushleft}


\begin{center}
  \textbf{\Large CÁC LINK TRELLO,GIT DÙNG TRONG DỰ ÁN}
\end{center}
\textbf{Link Trello:}
\url{https://trello.com/b/9ca0hRem}\\
\textbf{Link GIT:}
\url{https:github.com/UIT19521537/snake_Game-SS004.L16.CLC}

\begin{center}
    \textbf{\Large MÔ TẢ DỰ ÁN}
\end{center}
\textbf{1. Giới thiệu sơ lược các chức năng chính của game:}
\begin{itemize}
    \item[-]Như mọi người đã biết thì game con rắn(Snake game)là một trong những game gắn liền với tuổi thơ 9x hay đời đầu 10x,nó như là một kỉ niệm của mỗi người và ăn sâu vào tâm trí của mọi người.Với sự phát triển của công nghệ hiện nay thì game con rắn được cái tiến rất nhiều và có rất nhiều phiên bản.
    \item[-]Với game con rắn này nhóm chúng em đã sử dụng thêm thư viện SDL2 được viết bằng ngôn ngữ C nhưng hỗ trợ rất tốt với ngôn ngữ C++.SDL2 được biến đến là thư viện đa nền tảng để cung cấp khả năng truy cập cấp thấp vào phần cứng âm thanh,bàn phím,chuột,phím điều khiển và đồ họa thông qua OpenGL và Driect3D.Nó được sử dụng bởi phần mềm phát lại video,trình giả lập và các trò chơi phổ biến.
    \item[-]Class nhóm chúng em sử dụng là class game và chủ yếu sử dụng là thư viện SDL2 và thư viện mở rộng của nó là SD image.Các phương thức trong class lần lượt là:
    \begin{itemize}
        \item[•]Run:phương thức này dùng để khởi tạo cửa số của game đồng thời khởi tạo các Renderer để tạo nền vào các chi tiết game.
        \item[•]Getscore:phương thức này dùng để xuất điểm bạn đạt được.
        \item[•]Getsize:phương thức này dùng để trả lời về độ dài của rắn.
        \item[•]GameLoop:một phương thức đặc biệt giúp game chạy liên tục khi rắn vẫn đag còn sống.
        \item[•]PollEvents;phương thức này giúp xử lí dữ liệu từ con chuột,bàn phím để giúp con rắn chuyển động.
        \item[•]GrowBody:phương thức này giúp để cập nhật độ dài của con rắn.
        \item[•]Update:phương thức này giúp cập nhật khi con rắn di chuyển,ăn mồi,hay bất cứ thay đổi gì về mặt hình ảnh của game.
        \item[•]Render:phương thức này xuất các dữ liệu đã cập nhật lên màn hình.
        \item[•]Closs:đóng màn hình khi bạn thua và cập nhật điểm,size trên console
    \end{itemize}
\end{itemize}
\textbf{2. Điểm tâm đắc của nhóm em khi xây dựng game:}
\begin{itemize}
    \item [•]Thứ nhất:cảm ơn thầy đã cho chúng em cơ hội để tạo ra một game cho riêng mình và hơn nữa dạy cho chúng em hiểu biết nhiều về các kĩ năng cần thiết.
    \item[•]Thứ hai:nhóm chúng em cảm thấy phát hiện mang tính tự hào là sử dụng thư viện SDL2.Ngay từ những ngày đầu làm game thì nhóm chúng em đã tìm hiểu rất nhiều về đồ họa trong C++.Có những thư viện như Graphic.h hay OpenGL và cuối cùng nhóm chúng em đã chọn thư viện SDL2 để làm đồ án này.
    \newpage
    \item[•]Thứ ba:các phương thức trong class được viết khá dễ hiểu,hầu như chỉ cần hiểu một chút về C hoặc C++ là thì nhìn code là hiểu ngay.
    
\end{itemize}
\textbf{3. Điểm còn thiếu sót}

\begin{itemize}
    \item [•]Game vẫn chưa được hoàn thiện do nhóm chúng em chưa có thể nắm hết được các tính năng của SDL2.
    \item[•]Vì báo cáo của hai lớp khác nhau nên chưa làm đúng được quy trình làm theo.Nếu game còn update mới thì mời thầy xem lại báo cáo của bạn MSSV 19521537 vào ngày 15/12/2020 để biết thêm về những update mới.
    \item[•]Nhóm đã thành lập hơi trễ nên thời gian hiểu nhau chưa nhiều,có lúc cũng chưa làm việc ăn ý.
\end{itemize}
\newpage
\begin{center}
    \textbf{\Large ĐÁNH GIÁ VIỆC THỰC HIỆN HỢP ĐỒNG }
\end{center}
\textbf{I.Bảng đánh giá}\\
\begin{tabular}{|>{\raggedright\arraybackslash}p{4cm}
|>{\raggedright\arraybackslash}p{4cm}
|>{\raggedright\arraybackslash}p{4cm}
|>{\raggedright\arraybackslash}p{4cm}
|>{\raggedright\arraybackslash}p{4cm}|}
    \hline
        Đặc điểm
        & Hoàn thành tốt
        & Hoàn thành
        & Không hoàn thành\\
    \hline
        Thái độ làm việc
        & Hoàn thành tốt nhiệm vụ được giao
        & Hoàn thành nhiệm vụ với sự nhắc nhở
        & Không hoàn thành nhiệm vụ được giao\\
    \hline
        Quản lý thời gian
        & Hoàn thành nhiệm vụ trước thời hạn
        & Hoàn thành đúng thời hạn
        & Không hoàn thành đúng thời hạn
        \\
    \hline
        Giải quyết vấn đề phát sinh
        & Tích cực tìm kiếm giải pháp
        & Nhờ cậy hoặc là dùng sự hỗ trợ nhóm để giải quyết
        & Không tham gia vào vấn đề cần giải quyết
        \\
    \hline
        Nếu ý kiến
        & Sẵn sàng nêu ý kiến.
        & Đưa ra ý kiến một cách bị động
        & Không đưa ra ý kiến nào
        \\
    \hline
        Giữ liên lạc
        & Luôn giữ liên lạc
        & Liên lạc 1 cách thụ động
        & Khó khắn trong việc liên lạc
        \\
    \hline
        
\end{tabular}\\
\textbf{II.ĐÁNH GIÁ THEO TIÊU CHỈ CỦA TỪNG THÀNH VIÊN}
\begin{itemize}
    \item [•]\textbf{Nguyễn Duy Hoàng}
    \begin{itemize}
        \item [*]Thái độ làm việc:\textbf{Hoàn thành}
        \item[*]Quản lí thời gian:\textbf{Hoàn thành tốt}
        \item[*]Giải quyết vấn đề phát sinh:\textbf{Hoàn thành tốt}
        \item[*]Nêu ý kiến:\textbf{Hoàn thành}
        \item[*]Giữ liên lạc:\textbf{Hoàn thành tốt}
    \end{itemize}
    \item[•]\textbf{Trần Phi Long}
    \begin{itemize}
        \item [*]Thái độ làm việc:\textbf{Hoàn thành}
        \item[*]Quản lí thời gian:\textbf{Hoàn thành}
        \item[*]Giải quyết vấn đề phát sinh:\textbf{Hoàn thành}
        \item[*]Nêu ý kiến:\textbf{Chưa hoàn thành}
        \item[*]Giữ liên lạc:\textbf{Chưa hoàn thành}
    \end{itemize}
    \item[•]\textbf{Đào Văn Phú}
    \begin{itemize}
        \item [*]Thái độ làm việc:\textbf{Hoàn thành}
        \item[*]Quản lí thời gian:\textbf{Chưa hoàn thành}
        \item[*]Giải quyết vấn đề phát sinh:\textbf{Hoàn thành}
        \item[*]Nêu ý kiến:\textbf{Hoàn thành}
        \item[*]Giữ liên lạc:\textbf{Hoàn thành tốt}
        
    \end{itemize}
\end{itemize}
\textbf{NHẬN XÉT TỪNG THÀNH VIÊN}
\begin{itemize}
    \item [•]\textbf{Nguyễn Duy Hoàng}
    \begin{itemize}
        \item[-]Là nhóm trưởng phân công khá rõ ràng trong công việc, tích cực giúp đỡ các thành viên đồng thời là người đi đầu trong việc tìm hiểu các thư viện mới tiêu biểu là thư viện SDL2.
\item[-]Tuy vậy nhóm trưởng chưa tốt ở việc nêu ý kiến của mình dẫn tới việc các thành viên khó hiểu được ý muốn của nhóm trưởng. 
\item[-]Là người code chính trong team và là người hay sửa lỗi code của nhóm.

    \end{itemize}
    \newpage
    \item[•]\textbf{Trần Phi Long}
    \begin{itemize}
        \item[-]Tuy không nổi trội như nhóm trường những đóng góp khá nhiều trong việc code cũng như là tìm hiểu logic về game.
\item[-]Yếu điểm là hay mất liên lạc nên khó có thể hợp tác với team cộng thêm việc ít khi nêu ra ý kiến cá nhân khiến cho việc tìm lỗi sai khá khó. Cần cải thiện.

    \end{itemize}
    \item[•]\textbf{Đào Văn Phú}
    \begin{itemize}
        \item [-]Tester cũng như là người soạn thảo báo cáo.Là một người khá năng động trong công việc cũng như giữ liên lạc rất tốt nhưng yếu điểm là chưa quản lí được thời gian cũng như còn yếu về mặt code và tư duy.Nhưng sự nhiệt tình của bạn là một điều đáng ghi nhận.
    \end{itemize}
\end{itemize}
\begin{center}
    --------------------Hết------------------
\end{center}

\end{document}
